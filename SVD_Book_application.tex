
% Default to the notebook output style

    


% Inherit from the specified cell style.




    
\documentclass[11pt]{article}

    
    
    \usepackage[T1]{fontenc}
    % Nicer default font (+ math font) than Computer Modern for most use cases
    \usepackage{mathpazo}

    % Basic figure setup, for now with no caption control since it's done
    % automatically by Pandoc (which extracts ![](path) syntax from Markdown).
    \usepackage{graphicx}
    % We will generate all images so they have a width \maxwidth. This means
    % that they will get their normal width if they fit onto the page, but
    % are scaled down if they would overflow the margins.
    \makeatletter
    \def\maxwidth{\ifdim\Gin@nat@width>\linewidth\linewidth
    \else\Gin@nat@width\fi}
    \makeatother
    \let\Oldincludegraphics\includegraphics
    % Set max figure width to be 80% of text width, for now hardcoded.
    \renewcommand{\includegraphics}[1]{\Oldincludegraphics[width=.8\maxwidth]{#1}}
    % Ensure that by default, figures have no caption (until we provide a
    % proper Figure object with a Caption API and a way to capture that
    % in the conversion process - todo).
    \usepackage{caption}
    \DeclareCaptionLabelFormat{nolabel}{}
    \captionsetup{labelformat=nolabel}

    \usepackage{adjustbox} % Used to constrain images to a maximum size 
    \usepackage{xcolor} % Allow colors to be defined
    \usepackage{enumerate} % Needed for markdown enumerations to work
    \usepackage{geometry} % Used to adjust the document margins
    \usepackage{amsmath} % Equations
    \usepackage{amssymb} % Equations
    \usepackage{textcomp} % defines textquotesingle
    % Hack from http://tex.stackexchange.com/a/47451/13684:
    \AtBeginDocument{%
        \def\PYZsq{\textquotesingle}% Upright quotes in Pygmentized code
    }
    \usepackage{upquote} % Upright quotes for verbatim code
    \usepackage{eurosym} % defines \euro
    \usepackage[mathletters]{ucs} % Extended unicode (utf-8) support
    \usepackage[utf8x]{inputenc} % Allow utf-8 characters in the tex document
    \usepackage{fancyvrb} % verbatim replacement that allows latex
    \usepackage{grffile} % extends the file name processing of package graphics 
                         % to support a larger range 
    % The hyperref package gives us a pdf with properly built
    % internal navigation ('pdf bookmarks' for the table of contents,
    % internal cross-reference links, web links for URLs, etc.)
    \usepackage{hyperref}
    \usepackage{longtable} % longtable support required by pandoc >1.10
    \usepackage{booktabs}  % table support for pandoc > 1.12.2
    \usepackage[inline]{enumitem} % IRkernel/repr support (it uses the enumerate* environment)
    \usepackage[normalem]{ulem} % ulem is needed to support strikethroughs (\sout)
                                % normalem makes italics be italics, not underlines
    

    
    
    % Colors for the hyperref package
    \definecolor{urlcolor}{rgb}{0,.145,.698}
    \definecolor{linkcolor}{rgb}{.71,0.21,0.01}
    \definecolor{citecolor}{rgb}{.12,.54,.11}

    % ANSI colors
    \definecolor{ansi-black}{HTML}{3E424D}
    \definecolor{ansi-black-intense}{HTML}{282C36}
    \definecolor{ansi-red}{HTML}{E75C58}
    \definecolor{ansi-red-intense}{HTML}{B22B31}
    \definecolor{ansi-green}{HTML}{00A250}
    \definecolor{ansi-green-intense}{HTML}{007427}
    \definecolor{ansi-yellow}{HTML}{DDB62B}
    \definecolor{ansi-yellow-intense}{HTML}{B27D12}
    \definecolor{ansi-blue}{HTML}{208FFB}
    \definecolor{ansi-blue-intense}{HTML}{0065CA}
    \definecolor{ansi-magenta}{HTML}{D160C4}
    \definecolor{ansi-magenta-intense}{HTML}{A03196}
    \definecolor{ansi-cyan}{HTML}{60C6C8}
    \definecolor{ansi-cyan-intense}{HTML}{258F8F}
    \definecolor{ansi-white}{HTML}{C5C1B4}
    \definecolor{ansi-white-intense}{HTML}{A1A6B2}

    % commands and environments needed by pandoc snippets
    % extracted from the output of `pandoc -s`
    \providecommand{\tightlist}{%
      \setlength{\itemsep}{0pt}\setlength{\parskip}{0pt}}
    \DefineVerbatimEnvironment{Highlighting}{Verbatim}{commandchars=\\\{\}}
    % Add ',fontsize=\small' for more characters per line
    \newenvironment{Shaded}{}{}
    \newcommand{\KeywordTok}[1]{\textcolor[rgb]{0.00,0.44,0.13}{\textbf{{#1}}}}
    \newcommand{\DataTypeTok}[1]{\textcolor[rgb]{0.56,0.13,0.00}{{#1}}}
    \newcommand{\DecValTok}[1]{\textcolor[rgb]{0.25,0.63,0.44}{{#1}}}
    \newcommand{\BaseNTok}[1]{\textcolor[rgb]{0.25,0.63,0.44}{{#1}}}
    \newcommand{\FloatTok}[1]{\textcolor[rgb]{0.25,0.63,0.44}{{#1}}}
    \newcommand{\CharTok}[1]{\textcolor[rgb]{0.25,0.44,0.63}{{#1}}}
    \newcommand{\StringTok}[1]{\textcolor[rgb]{0.25,0.44,0.63}{{#1}}}
    \newcommand{\CommentTok}[1]{\textcolor[rgb]{0.38,0.63,0.69}{\textit{{#1}}}}
    \newcommand{\OtherTok}[1]{\textcolor[rgb]{0.00,0.44,0.13}{{#1}}}
    \newcommand{\AlertTok}[1]{\textcolor[rgb]{1.00,0.00,0.00}{\textbf{{#1}}}}
    \newcommand{\FunctionTok}[1]{\textcolor[rgb]{0.02,0.16,0.49}{{#1}}}
    \newcommand{\RegionMarkerTok}[1]{{#1}}
    \newcommand{\ErrorTok}[1]{\textcolor[rgb]{1.00,0.00,0.00}{\textbf{{#1}}}}
    \newcommand{\NormalTok}[1]{{#1}}
    
    % Additional commands for more recent versions of Pandoc
    \newcommand{\ConstantTok}[1]{\textcolor[rgb]{0.53,0.00,0.00}{{#1}}}
    \newcommand{\SpecialCharTok}[1]{\textcolor[rgb]{0.25,0.44,0.63}{{#1}}}
    \newcommand{\VerbatimStringTok}[1]{\textcolor[rgb]{0.25,0.44,0.63}{{#1}}}
    \newcommand{\SpecialStringTok}[1]{\textcolor[rgb]{0.73,0.40,0.53}{{#1}}}
    \newcommand{\ImportTok}[1]{{#1}}
    \newcommand{\DocumentationTok}[1]{\textcolor[rgb]{0.73,0.13,0.13}{\textit{{#1}}}}
    \newcommand{\AnnotationTok}[1]{\textcolor[rgb]{0.38,0.63,0.69}{\textbf{\textit{{#1}}}}}
    \newcommand{\CommentVarTok}[1]{\textcolor[rgb]{0.38,0.63,0.69}{\textbf{\textit{{#1}}}}}
    \newcommand{\VariableTok}[1]{\textcolor[rgb]{0.10,0.09,0.49}{{#1}}}
    \newcommand{\ControlFlowTok}[1]{\textcolor[rgb]{0.00,0.44,0.13}{\textbf{{#1}}}}
    \newcommand{\OperatorTok}[1]{\textcolor[rgb]{0.40,0.40,0.40}{{#1}}}
    \newcommand{\BuiltInTok}[1]{{#1}}
    \newcommand{\ExtensionTok}[1]{{#1}}
    \newcommand{\PreprocessorTok}[1]{\textcolor[rgb]{0.74,0.48,0.00}{{#1}}}
    \newcommand{\AttributeTok}[1]{\textcolor[rgb]{0.49,0.56,0.16}{{#1}}}
    \newcommand{\InformationTok}[1]{\textcolor[rgb]{0.38,0.63,0.69}{\textbf{\textit{{#1}}}}}
    \newcommand{\WarningTok}[1]{\textcolor[rgb]{0.38,0.63,0.69}{\textbf{\textit{{#1}}}}}
    
    
    % Define a nice break command that doesn't care if a line doesn't already
    % exist.
    \def\br{\hspace*{\fill} \\* }
    % Math Jax compatability definitions
    \def\gt{>}
    \def\lt{<}
    % Document parameters
    \title{Dimensional Reduction: Code Example}
    \author{Fernando Tocantins}
    \date{January, 2020}
    
    
    

    % Pygments definitions
    
\makeatletter
\def\PY@reset{\let\PY@it=\relax \let\PY@bf=\relax%
    \let\PY@ul=\relax \let\PY@tc=\relax%
    \let\PY@bc=\relax \let\PY@ff=\relax}
\def\PY@tok#1{\csname PY@tok@#1\endcsname}
\def\PY@toks#1+{\ifx\relax#1\empty\else%
    \PY@tok{#1}\expandafter\PY@toks\fi}
\def\PY@do#1{\PY@bc{\PY@tc{\PY@ul{%
    \PY@it{\PY@bf{\PY@ff{#1}}}}}}}
\def\PY#1#2{\PY@reset\PY@toks#1+\relax+\PY@do{#2}}

\expandafter\def\csname PY@tok@w\endcsname{\def\PY@tc##1{\textcolor[rgb]{0.73,0.73,0.73}{##1}}}
\expandafter\def\csname PY@tok@c\endcsname{\let\PY@it=\textit\def\PY@tc##1{\textcolor[rgb]{0.25,0.50,0.50}{##1}}}
\expandafter\def\csname PY@tok@cp\endcsname{\def\PY@tc##1{\textcolor[rgb]{0.74,0.48,0.00}{##1}}}
\expandafter\def\csname PY@tok@k\endcsname{\let\PY@bf=\textbf\def\PY@tc##1{\textcolor[rgb]{0.00,0.50,0.00}{##1}}}
\expandafter\def\csname PY@tok@kp\endcsname{\def\PY@tc##1{\textcolor[rgb]{0.00,0.50,0.00}{##1}}}
\expandafter\def\csname PY@tok@kt\endcsname{\def\PY@tc##1{\textcolor[rgb]{0.69,0.00,0.25}{##1}}}
\expandafter\def\csname PY@tok@o\endcsname{\def\PY@tc##1{\textcolor[rgb]{0.40,0.40,0.40}{##1}}}
\expandafter\def\csname PY@tok@ow\endcsname{\let\PY@bf=\textbf\def\PY@tc##1{\textcolor[rgb]{0.67,0.13,1.00}{##1}}}
\expandafter\def\csname PY@tok@nb\endcsname{\def\PY@tc##1{\textcolor[rgb]{0.00,0.50,0.00}{##1}}}
\expandafter\def\csname PY@tok@nf\endcsname{\def\PY@tc##1{\textcolor[rgb]{0.00,0.00,1.00}{##1}}}
\expandafter\def\csname PY@tok@nc\endcsname{\let\PY@bf=\textbf\def\PY@tc##1{\textcolor[rgb]{0.00,0.00,1.00}{##1}}}
\expandafter\def\csname PY@tok@nn\endcsname{\let\PY@bf=\textbf\def\PY@tc##1{\textcolor[rgb]{0.00,0.00,1.00}{##1}}}
\expandafter\def\csname PY@tok@ne\endcsname{\let\PY@bf=\textbf\def\PY@tc##1{\textcolor[rgb]{0.82,0.25,0.23}{##1}}}
\expandafter\def\csname PY@tok@nv\endcsname{\def\PY@tc##1{\textcolor[rgb]{0.10,0.09,0.49}{##1}}}
\expandafter\def\csname PY@tok@no\endcsname{\def\PY@tc##1{\textcolor[rgb]{0.53,0.00,0.00}{##1}}}
\expandafter\def\csname PY@tok@nl\endcsname{\def\PY@tc##1{\textcolor[rgb]{0.63,0.63,0.00}{##1}}}
\expandafter\def\csname PY@tok@ni\endcsname{\let\PY@bf=\textbf\def\PY@tc##1{\textcolor[rgb]{0.60,0.60,0.60}{##1}}}
\expandafter\def\csname PY@tok@na\endcsname{\def\PY@tc##1{\textcolor[rgb]{0.49,0.56,0.16}{##1}}}
\expandafter\def\csname PY@tok@nt\endcsname{\let\PY@bf=\textbf\def\PY@tc##1{\textcolor[rgb]{0.00,0.50,0.00}{##1}}}
\expandafter\def\csname PY@tok@nd\endcsname{\def\PY@tc##1{\textcolor[rgb]{0.67,0.13,1.00}{##1}}}
\expandafter\def\csname PY@tok@s\endcsname{\def\PY@tc##1{\textcolor[rgb]{0.73,0.13,0.13}{##1}}}
\expandafter\def\csname PY@tok@sd\endcsname{\let\PY@it=\textit\def\PY@tc##1{\textcolor[rgb]{0.73,0.13,0.13}{##1}}}
\expandafter\def\csname PY@tok@si\endcsname{\let\PY@bf=\textbf\def\PY@tc##1{\textcolor[rgb]{0.73,0.40,0.53}{##1}}}
\expandafter\def\csname PY@tok@se\endcsname{\let\PY@bf=\textbf\def\PY@tc##1{\textcolor[rgb]{0.73,0.40,0.13}{##1}}}
\expandafter\def\csname PY@tok@sr\endcsname{\def\PY@tc##1{\textcolor[rgb]{0.73,0.40,0.53}{##1}}}
\expandafter\def\csname PY@tok@ss\endcsname{\def\PY@tc##1{\textcolor[rgb]{0.10,0.09,0.49}{##1}}}
\expandafter\def\csname PY@tok@sx\endcsname{\def\PY@tc##1{\textcolor[rgb]{0.00,0.50,0.00}{##1}}}
\expandafter\def\csname PY@tok@m\endcsname{\def\PY@tc##1{\textcolor[rgb]{0.40,0.40,0.40}{##1}}}
\expandafter\def\csname PY@tok@gh\endcsname{\let\PY@bf=\textbf\def\PY@tc##1{\textcolor[rgb]{0.00,0.00,0.50}{##1}}}
\expandafter\def\csname PY@tok@gu\endcsname{\let\PY@bf=\textbf\def\PY@tc##1{\textcolor[rgb]{0.50,0.00,0.50}{##1}}}
\expandafter\def\csname PY@tok@gd\endcsname{\def\PY@tc##1{\textcolor[rgb]{0.63,0.00,0.00}{##1}}}
\expandafter\def\csname PY@tok@gi\endcsname{\def\PY@tc##1{\textcolor[rgb]{0.00,0.63,0.00}{##1}}}
\expandafter\def\csname PY@tok@gr\endcsname{\def\PY@tc##1{\textcolor[rgb]{1.00,0.00,0.00}{##1}}}
\expandafter\def\csname PY@tok@ge\endcsname{\let\PY@it=\textit}
\expandafter\def\csname PY@tok@gs\endcsname{\let\PY@bf=\textbf}
\expandafter\def\csname PY@tok@gp\endcsname{\let\PY@bf=\textbf\def\PY@tc##1{\textcolor[rgb]{0.00,0.00,0.50}{##1}}}
\expandafter\def\csname PY@tok@go\endcsname{\def\PY@tc##1{\textcolor[rgb]{0.53,0.53,0.53}{##1}}}
\expandafter\def\csname PY@tok@gt\endcsname{\def\PY@tc##1{\textcolor[rgb]{0.00,0.27,0.87}{##1}}}
\expandafter\def\csname PY@tok@err\endcsname{\def\PY@bc##1{\setlength{\fboxsep}{0pt}\fcolorbox[rgb]{1.00,0.00,0.00}{1,1,1}{\strut ##1}}}
\expandafter\def\csname PY@tok@kc\endcsname{\let\PY@bf=\textbf\def\PY@tc##1{\textcolor[rgb]{0.00,0.50,0.00}{##1}}}
\expandafter\def\csname PY@tok@kd\endcsname{\let\PY@bf=\textbf\def\PY@tc##1{\textcolor[rgb]{0.00,0.50,0.00}{##1}}}
\expandafter\def\csname PY@tok@kn\endcsname{\let\PY@bf=\textbf\def\PY@tc##1{\textcolor[rgb]{0.00,0.50,0.00}{##1}}}
\expandafter\def\csname PY@tok@kr\endcsname{\let\PY@bf=\textbf\def\PY@tc##1{\textcolor[rgb]{0.00,0.50,0.00}{##1}}}
\expandafter\def\csname PY@tok@bp\endcsname{\def\PY@tc##1{\textcolor[rgb]{0.00,0.50,0.00}{##1}}}
\expandafter\def\csname PY@tok@fm\endcsname{\def\PY@tc##1{\textcolor[rgb]{0.00,0.00,1.00}{##1}}}
\expandafter\def\csname PY@tok@vc\endcsname{\def\PY@tc##1{\textcolor[rgb]{0.10,0.09,0.49}{##1}}}
\expandafter\def\csname PY@tok@vg\endcsname{\def\PY@tc##1{\textcolor[rgb]{0.10,0.09,0.49}{##1}}}
\expandafter\def\csname PY@tok@vi\endcsname{\def\PY@tc##1{\textcolor[rgb]{0.10,0.09,0.49}{##1}}}
\expandafter\def\csname PY@tok@vm\endcsname{\def\PY@tc##1{\textcolor[rgb]{0.10,0.09,0.49}{##1}}}
\expandafter\def\csname PY@tok@sa\endcsname{\def\PY@tc##1{\textcolor[rgb]{0.73,0.13,0.13}{##1}}}
\expandafter\def\csname PY@tok@sb\endcsname{\def\PY@tc##1{\textcolor[rgb]{0.73,0.13,0.13}{##1}}}
\expandafter\def\csname PY@tok@sc\endcsname{\def\PY@tc##1{\textcolor[rgb]{0.73,0.13,0.13}{##1}}}
\expandafter\def\csname PY@tok@dl\endcsname{\def\PY@tc##1{\textcolor[rgb]{0.73,0.13,0.13}{##1}}}
\expandafter\def\csname PY@tok@s2\endcsname{\def\PY@tc##1{\textcolor[rgb]{0.73,0.13,0.13}{##1}}}
\expandafter\def\csname PY@tok@sh\endcsname{\def\PY@tc##1{\textcolor[rgb]{0.73,0.13,0.13}{##1}}}
\expandafter\def\csname PY@tok@s1\endcsname{\def\PY@tc##1{\textcolor[rgb]{0.73,0.13,0.13}{##1}}}
\expandafter\def\csname PY@tok@mb\endcsname{\def\PY@tc##1{\textcolor[rgb]{0.40,0.40,0.40}{##1}}}
\expandafter\def\csname PY@tok@mf\endcsname{\def\PY@tc##1{\textcolor[rgb]{0.40,0.40,0.40}{##1}}}
\expandafter\def\csname PY@tok@mh\endcsname{\def\PY@tc##1{\textcolor[rgb]{0.40,0.40,0.40}{##1}}}
\expandafter\def\csname PY@tok@mi\endcsname{\def\PY@tc##1{\textcolor[rgb]{0.40,0.40,0.40}{##1}}}
\expandafter\def\csname PY@tok@il\endcsname{\def\PY@tc##1{\textcolor[rgb]{0.40,0.40,0.40}{##1}}}
\expandafter\def\csname PY@tok@mo\endcsname{\def\PY@tc##1{\textcolor[rgb]{0.40,0.40,0.40}{##1}}}
\expandafter\def\csname PY@tok@ch\endcsname{\let\PY@it=\textit\def\PY@tc##1{\textcolor[rgb]{0.25,0.50,0.50}{##1}}}
\expandafter\def\csname PY@tok@cm\endcsname{\let\PY@it=\textit\def\PY@tc##1{\textcolor[rgb]{0.25,0.50,0.50}{##1}}}
\expandafter\def\csname PY@tok@cpf\endcsname{\let\PY@it=\textit\def\PY@tc##1{\textcolor[rgb]{0.25,0.50,0.50}{##1}}}
\expandafter\def\csname PY@tok@c1\endcsname{\let\PY@it=\textit\def\PY@tc##1{\textcolor[rgb]{0.25,0.50,0.50}{##1}}}
\expandafter\def\csname PY@tok@cs\endcsname{\let\PY@it=\textit\def\PY@tc##1{\textcolor[rgb]{0.25,0.50,0.50}{##1}}}

\def\PYZbs{\char`\\}
\def\PYZus{\char`\_}
\def\PYZob{\char`\{}
\def\PYZcb{\char`\}}
\def\PYZca{\char`\^}
\def\PYZam{\char`\&}
\def\PYZlt{\char`\<}
\def\PYZgt{\char`\>}
\def\PYZsh{\char`\#}
\def\PYZpc{\char`\%}
\def\PYZdl{\char`\$}
\def\PYZhy{\char`\-}
\def\PYZsq{\char`\'}
\def\PYZdq{\char`\"}
\def\PYZti{\char`\~}
% for compatibility with earlier versions
\def\PYZat{@}
\def\PYZlb{[}
\def\PYZrb{]}
\makeatother


    % Exact colors from NB
    \definecolor{incolor}{rgb}{0.0, 0.0, 0.5}
    \definecolor{outcolor}{rgb}{0.545, 0.0, 0.0}



    
    % Prevent overflowing lines due to hard-to-break entities
    \sloppy 
    % Setup hyperref package
    \hypersetup{
      breaklinks=true,  % so long urls are correctly broken across lines
      colorlinks=true,
      urlcolor=urlcolor,
      linkcolor=linkcolor,
      citecolor=citecolor,
      }
    % Slightly bigger margins than the latex defaults
    
    \geometry{verbose,tmargin=1in,bmargin=1in,lmargin=1in,rmargin=1in}
    
    

    \begin{document}
    
    
    \maketitle
    
    

    
    \section{SVD application in text
books}\label{svd-application-in-text-books}

    \subparagraph{Libraries}\label{libraries}

\begin{itemize}
\tightlist
\item
  Pandas https://pandas.pydata.org/, a tool for easy data structring,
  reading and analyses. It will be used for imprting and reading the
  document
\item
  MatplotLib https://matplotlib.org/, for graph plotting.
\item
  Numpy https://numpy.org/, almost a standard library for scientific
  computing, has many resources for data manipulation.
\item
  NLTK https://www.nltk.org/, a tool for natural language processing
  NLTK for building Python programs to work with human language data. It
  provides easy-to-use interfaces of text processing libraries for
  classification, tokenization, semantic reasoning and many others.
\item
  Urllib https://docs.python.org/3/library/urllib.html, is a package
  that collects several modules for working with URLs.
\end{itemize}

    \begin{quote}
This code example will show how Singular value Decomposition can be used
for create a vector space of words based on a co-occurance matrix from
some sentences of real books.
\end{quote}

    \begin{Verbatim}[commandchars=\\\{\}]
{\color{incolor}In [{\color{incolor}23}]:} \PY{c+c1}{\PYZsh{} Importing the libraries }
         \PY{k+kn}{import} \PY{n+nn}{pandas} \PY{k}{as} \PY{n+nn}{pd}
         \PY{k+kn}{import} \PY{n+nn}{numpy} \PY{k}{as} \PY{n+nn}{np}
         \PY{k+kn}{import} \PY{n+nn}{matplotlib}\PY{n+nn}{.}\PY{n+nn}{pyplot} \PY{k}{as} \PY{n+nn}{plt}
         \PY{k+kn}{import} \PY{n+nn}{nltk}
         \PY{k+kn}{from} \PY{n+nn}{nltk} \PY{k}{import} \PY{n}{bigrams}
         \PY{k+kn}{import} \PY{n+nn}{itertools}
         \PY{k+kn}{from} \PY{n+nn}{urllib} \PY{k}{import} \PY{n}{request}
\end{Verbatim}


    \begin{quote}
Project Gutenberg is a library of over 60,000 free eBooks. The request
below will download the data for our code and store in raw. The book
selected was "Adventures of Sherlock Holmes, by A. Conan Doyle"
\end{quote}

    \begin{Verbatim}[commandchars=\\\{\}]
{\color{incolor}In [{\color{incolor}7}]:} \PY{n}{url} \PY{o}{=} \PY{l+s+s2}{\PYZdq{}}\PY{l+s+s2}{http://www.gutenberg.org/files/48320/48320\PYZhy{}0.txt}\PY{l+s+s2}{\PYZdq{}}
        \PY{n}{response} \PY{o}{=} \PY{n}{request}\PY{o}{.}\PY{n}{urlopen}\PY{p}{(}\PY{n}{url}\PY{p}{)}
        \PY{n}{raw} \PY{o}{=} \PY{n}{response}\PY{o}{.}\PY{n}{read}\PY{p}{(}\PY{p}{)}\PY{o}{.}\PY{n}{decode}\PY{p}{(}\PY{l+s+s1}{\PYZsq{}}\PY{l+s+s1}{utf8}\PY{l+s+s1}{\PYZsq{}}\PY{p}{)}
\end{Verbatim}


    \begin{Verbatim}[commandchars=\\\{\}]
{\color{incolor}In [{\color{incolor}12}]:} \PY{n+nb}{print}\PY{p}{(}\PY{n+nb}{type}\PY{p}{(}\PY{n}{raw}\PY{p}{)}\PY{p}{,} \PY{l+s+s2}{\PYZdq{}}\PY{l+s+se}{\PYZbs{}n}\PY{l+s+s2}{\PYZdq{}}\PY{p}{)}
         \PY{n+nb}{print}\PY{p}{(}\PY{l+s+s2}{\PYZdq{}}\PY{l+s+s2}{lenght: }\PY{l+s+s2}{\PYZdq{}}\PY{p}{,}\PY{n+nb}{len}\PY{p}{(}\PY{n}{raw}\PY{p}{)}\PY{p}{,} \PY{l+s+s2}{\PYZdq{}}\PY{l+s+se}{\PYZbs{}n}\PY{l+s+s2}{\PYZdq{}}\PY{p}{)}
         \PY{n+nb}{print}\PY{p}{(}\PY{n}{raw}\PY{p}{[}\PY{p}{:}\PY{l+m+mi}{75}\PY{p}{]}\PY{p}{)}
\end{Verbatim}


    \begin{Verbatim}[commandchars=\\\{\}]
<class 'str'> 

lenght:  608162 

Project Gutenberg's Adventures of Sherlock Holmes, by A. Conan Doyle

Th

    \end{Verbatim}

    \begin{Verbatim}[commandchars=\\\{\}]
{\color{incolor}In [{\color{incolor}13}]:} \PY{c+c1}{\PYZsh{} Selecting the corpus data (just the book content) \PYZhy{}\PYZhy{}\PYZhy{} Different for every book}
         \PY{c+c1}{\PYZsh{} The book \PYZdq{}Adventure I\PYZdq{}  }
         
         \PY{n}{raw}\PY{o}{.}\PY{n}{find}\PY{p}{(}\PY{l+s+s2}{\PYZdq{}}\PY{l+s+s2}{Adventure I}\PY{l+s+s2}{\PYZdq{}}\PY{p}{)} \PY{c+c1}{\PYZsh{} search from the beginning}
         \PY{n}{raw}\PY{o}{.}\PY{n}{rfind}\PY{p}{(}\PY{l+s+s2}{\PYZdq{}}\PY{l+s+s2}{Adventure I}\PY{l+s+s2}{\PYZdq{}}\PY{p}{)} \PY{c+c1}{\PYZsh{} search in from the end}
         \PY{n}{raw} \PY{o}{=} \PY{n}{raw}\PY{p}{[}\PY{l+m+mi}{3248}\PY{p}{:}\PY{l+m+mi}{381303}\PY{p}{]}
         
         \PY{c+c1}{\PYZsh{} Selecting just a fraction of the entire dataset}
         \PY{n}{short\PYZus{}raw} \PY{o}{=} \PY{n}{raw}\PY{p}{[}\PY{l+m+mi}{1024}\PY{p}{:}\PY{l+m+mi}{200000}\PY{p}{]}
\end{Verbatim}


    \begin{Verbatim}[commandchars=\\\{\}]
{\color{incolor}In [{\color{incolor}25}]:} \PY{c+c1}{\PYZsh{} Tokenize without symbols. Here we take out the symbols from the text}
         \PY{k+kn}{from} \PY{n+nn}{nltk}\PY{n+nn}{.}\PY{n+nn}{tokenize} \PY{k}{import} \PY{n}{RegexpTokenizer}
         \PY{n}{tokenizer} \PY{o}{=} \PY{n}{RegexpTokenizer}\PY{p}{(}\PY{l+s+sa}{r}\PY{l+s+s1}{\PYZsq{}}\PY{l+s+s1}{\PYZbs{}}\PY{l+s+s1}{w+}\PY{l+s+s1}{\PYZsq{}}\PY{p}{)}
         \PY{n}{corpus} \PY{o}{=} \PY{n}{tokenizer}\PY{o}{.}\PY{n}{tokenize}\PY{p}{(}\PY{n}{short\PYZus{}raw}\PY{o}{.}\PY{n}{lower}\PY{p}{(}\PY{p}{)}\PY{p}{)}
         
         \PY{n+nb}{print}\PY{p}{(}\PY{n}{corpus}\PY{p}{[}\PY{p}{:}\PY{l+m+mi}{100}\PY{p}{]}\PY{p}{)}
\end{Verbatim}


    \begin{Verbatim}[commandchars=\\\{\}]
['ot', 'be', 'more', 'disturbing', 'than', 'a', 'strong', 'emotion', 'in', 'a', 'nature', 'such', 'as', 'his', 'and', 'yet', 'there', 'was', 'but', 'one', 'woman', 'to', 'him', 'and', 'that', 'woman', 'was', 'the', 'late', 'irene', 'adler', 'of', 'dubious', 'and', 'questionable', 'memory', 'i', 'had', 'seen', 'little', 'of', 'holmes', 'lately', 'my', 'marriage', 'had', 'drifted', 'us', 'away', 'from', 'each', 'other', 'my', 'own', 'complete', 'happiness', 'and', 'the', 'home', 'centred', 'interests', 'which', 'rise', 'up', 'around', 'the', 'man', 'who', 'first', 'finds', 'himself', 'master', 'of', 'his', 'own', 'establishment', 'were', 'sufficient', 'to', 'absorb', 'all', 'my', 'attention', 'while', 'holmes', 'who', 'loathed', 'every', 'form', 'of', 'society', 'with', 'his', 'whole', 'bohemian', 'soul', 'remained', 'in', 'our', 'lodgings']

    \end{Verbatim}

    \begin{Verbatim}[commandchars=\\\{\}]
{\color{incolor}In [{\color{incolor}26}]:} \PY{c+c1}{\PYZsh{} Stopwords removal. The stop words removal is the removal of the most common words in a language, such as \PYZdq{}the\PYZdq{}, \PYZdq{}as\PYZdq{}, \PYZdq{}a\PYZdq{}.}
         \PY{k+kn}{from} \PY{n+nn}{nltk}\PY{n+nn}{.}\PY{n+nn}{corpus} \PY{k}{import} \PY{n}{stopwords} 
         \PY{n}{stop\PYZus{}words} \PY{o}{=} \PY{n+nb}{set}\PY{p}{(}\PY{n}{stopwords}\PY{o}{.}\PY{n}{words}\PY{p}{(}\PY{l+s+s1}{\PYZsq{}}\PY{l+s+s1}{english}\PY{l+s+s1}{\PYZsq{}}\PY{p}{)}\PY{p}{)} 
         \PY{n}{filtered\PYZus{}sentence} \PY{o}{=} \PY{p}{[}\PY{n}{w} \PY{k}{for} \PY{n}{w} \PY{o+ow}{in} \PY{n}{corpus} \PY{k}{if} \PY{o+ow}{not} \PY{n}{w} \PY{o+ow}{in} \PY{n}{stop\PYZus{}words}\PY{p}{]} 
         
         \PY{n+nb}{len}\PY{p}{(}\PY{n}{filtered\PYZus{}sentence}\PY{p}{)}
         \PY{n}{corpus} \PY{o}{=} \PY{n}{filtered\PYZus{}sentence}
         \PY{n+nb}{print}\PY{p}{(}\PY{n}{corpus}\PY{p}{[}\PY{p}{:}\PY{l+m+mi}{100}\PY{p}{]}\PY{p}{)}
\end{Verbatim}


    \begin{Verbatim}[commandchars=\\\{\}]
['ot', 'disturbing', 'strong', 'emotion', 'nature', 'yet', 'one', 'woman', 'woman', 'late', 'irene', 'adler', 'dubious', 'questionable', 'memory', 'seen', 'little', 'holmes', 'lately', 'marriage', 'drifted', 'us', 'away', 'complete', 'happiness', 'home', 'centred', 'interests', 'rise', 'around', 'man', 'first', 'finds', 'master', 'establishment', 'sufficient', 'absorb', 'attention', 'holmes', 'loathed', 'every', 'form', 'society', 'whole', 'bohemian', 'soul', 'remained', 'lodgings', 'baker', 'street', 'buried', 'among', 'old', 'books', 'alternating', 'week', 'week', 'cocaine', 'ambition', 'drowsiness', 'drug', 'fierce', 'energy', 'keen', 'nature', 'still', 'ever', 'deeply', 'attracted', 'study', 'crime', 'occupied', 'immense', 'faculties', 'extraordinary', 'powers', 'observation', 'following', 'clues', 'clearing', 'mysteries', 'abandoned', 'hopeless', 'official', 'police', 'time', 'time', 'heard', 'vague', 'account', 'doings', 'summons', 'odessa', 'case', 'trepoff', 'murder', 'clearing', 'singular', 'tragedy', 'atkinson']

    \end{Verbatim}

    \begin{Verbatim}[commandchars=\\\{\}]
{\color{incolor}In [{\color{incolor}20}]:} \PY{c+c1}{\PYZsh{}\PYZsh{} Bigram implementation. The bigram implementation is the creation of a set of pairs of words that appears next to each other}
         \PY{n}{vocab} \PY{o}{=} \PY{n+nb}{set}\PY{p}{(}\PY{n}{corpus}\PY{p}{)}
         \PY{n}{vocab} \PY{o}{=} \PY{n+nb}{list}\PY{p}{(}\PY{n}{vocab}\PY{p}{)}
         \PY{n}{vocab\PYZus{}index} \PY{o}{=} \PY{p}{\PYZob{}}\PY{n}{word}\PY{p}{:} \PY{n}{i} \PY{k}{for} \PY{n}{i}\PY{p}{,} \PY{n}{word} \PY{o+ow}{in} \PY{n+nb}{enumerate}\PY{p}{(}\PY{n}{vocab}\PY{p}{)}\PY{p}{\PYZcb{}}
          
         \PY{c+c1}{\PYZsh{} Create bigrams from all words in corpus}
         \PY{n}{bi\PYZus{}grams} \PY{o}{=} \PY{n+nb}{list}\PY{p}{(}\PY{n}{bigrams}\PY{p}{(}\PY{n}{corpus}\PY{p}{)}\PY{p}{)}
          
         \PY{c+c1}{\PYZsh{} Frequency distribution of bigrams ((word1, word2), num\PYZus{}occurrences)}
         \PY{n}{bigram\PYZus{}freq} \PY{o}{=} \PY{n}{nltk}\PY{o}{.}\PY{n}{FreqDist}\PY{p}{(}\PY{n}{bi\PYZus{}grams}\PY{p}{)}\PY{o}{.}\PY{n}{most\PYZus{}common}\PY{p}{(}\PY{n+nb}{len}\PY{p}{(}\PY{n}{bi\PYZus{}grams}\PY{p}{)}\PY{p}{)}
          
         \PY{c+c1}{\PYZsh{} Initialise co\PYZhy{}occurrence matrix}
         \PY{c+c1}{\PYZsh{} co\PYZus{}occurrence\PYZus{}matrix[current][previous]}
         \PY{n}{co\PYZus{}occurrence\PYZus{}matrix} \PY{o}{=} \PY{n}{np}\PY{o}{.}\PY{n}{zeros}\PY{p}{(}\PY{p}{(}\PY{n+nb}{len}\PY{p}{(}\PY{n}{vocab}\PY{p}{)}\PY{p}{,} \PY{n+nb}{len}\PY{p}{(}\PY{n}{vocab}\PY{p}{)}\PY{p}{)}\PY{p}{)}
          
         \PY{c+c1}{\PYZsh{} Loop through the bigrams taking the current and previous word,}
         \PY{c+c1}{\PYZsh{} and the number of occurrences of the bigram.}
         \PY{k}{for} \PY{n}{bigram} \PY{o+ow}{in} \PY{n}{bigram\PYZus{}freq}\PY{p}{:}
             \PY{n}{current} \PY{o}{=} \PY{n}{bigram}\PY{p}{[}\PY{l+m+mi}{0}\PY{p}{]}\PY{p}{[}\PY{l+m+mi}{1}\PY{p}{]}
             \PY{n}{previous} \PY{o}{=} \PY{n}{bigram}\PY{p}{[}\PY{l+m+mi}{0}\PY{p}{]}\PY{p}{[}\PY{l+m+mi}{0}\PY{p}{]}
             \PY{n}{count} \PY{o}{=} \PY{n}{bigram}\PY{p}{[}\PY{l+m+mi}{1}\PY{p}{]}
             \PY{n}{pos\PYZus{}current} \PY{o}{=} \PY{n}{vocab\PYZus{}index}\PY{p}{[}\PY{n}{current}\PY{p}{]}
             \PY{n}{pos\PYZus{}previous} \PY{o}{=} \PY{n}{vocab\PYZus{}index}\PY{p}{[}\PY{n}{previous}\PY{p}{]}
             \PY{n}{co\PYZus{}occurrence\PYZus{}matrix}\PY{p}{[}\PY{n}{pos\PYZus{}current}\PY{p}{]}\PY{p}{[}\PY{n}{pos\PYZus{}previous}\PY{p}{]} \PY{o}{=} \PY{n}{count}
             
         \PY{c+c1}{\PYZsh{} The co\PYZhy{}occurance matrix is the matrix of all intersections of occurance between words   }
         \PY{n}{co\PYZus{}occurrence\PYZus{}matrix\PYZus{}complete} \PY{o}{=} \PY{n}{np}\PY{o}{.}\PY{n}{matrix}\PY{p}{(}\PY{n}{co\PYZus{}occurrence\PYZus{}matrix}\PY{p}{)}
          
         \PY{n+nb}{print}\PY{p}{(}\PY{l+s+s2}{\PYZdq{}}\PY{l+s+s2}{lenght: }\PY{l+s+s2}{\PYZdq{}}\PY{p}{,}\PY{n+nb}{len}\PY{p}{(}\PY{n}{co\PYZus{}occurrence\PYZus{}matrix\PYZus{}complete}\PY{p}{)}\PY{p}{,} \PY{l+s+s2}{\PYZdq{}}\PY{l+s+se}{\PYZbs{}n}\PY{l+s+s2}{\PYZdq{}}\PY{p}{)}
         \PY{n+nb}{print}\PY{p}{(}\PY{n}{co\PYZus{}occurrence\PYZus{}matrix\PYZus{}complete}\PY{p}{)}
\end{Verbatim}


    \begin{Verbatim}[commandchars=\\\{\}]
lenght:  4403 

[[0. 0. 0. {\ldots} 0. 0. 0.]
 [0. 0. 0. {\ldots} 0. 0. 0.]
 [0. 0. 0. {\ldots} 0. 0. 0.]
 {\ldots}
 [0. 0. 0. {\ldots} 0. 0. 0.]
 [0. 0. 0. {\ldots} 0. 0. 1.]
 [0. 0. 0. {\ldots} 0. 0. 0.]]

    \end{Verbatim}

    \begin{Verbatim}[commandchars=\\\{\}]
{\color{incolor}In [{\color{incolor}29}]:} \PY{c+c1}{\PYZsh{} Creating a dataframe for the co\PYZhy{}occurance matrix}
         \PY{n}{matrix}\PY{p}{,} \PY{n}{vocab\PYZus{}index} \PY{o}{=} \PY{n}{co\PYZus{}occurrence\PYZus{}matrix\PYZus{}complete}\PY{p}{,} \PY{n}{vocab\PYZus{}index} 
         \PY{n}{data\PYZus{}matrix} \PY{o}{=} \PY{n}{pd}\PY{o}{.}\PY{n}{DataFrame}\PY{p}{(}\PY{n}{matrix}\PY{p}{,} \PY{n}{index}\PY{o}{=}\PY{n}{vocab\PYZus{}index}\PY{p}{,} \PY{n}{columns}\PY{o}{=}\PY{n}{vocab\PYZus{}index}\PY{p}{)}
         \PY{n+nb}{print}\PY{p}{(}\PY{n}{data\PYZus{}matrix}\PY{o}{.}\PY{n}{iloc}\PY{p}{[}\PY{p}{:}\PY{l+m+mi}{10}\PY{p}{,}\PY{p}{:}\PY{l+m+mi}{10}\PY{p}{]}\PY{p}{)}
\end{Verbatim}


    \begin{Verbatim}[commandchars=\\\{\}]
             couch  plunging  tide  violent  stethoscope  chairs  work  \textbackslash{}
couch          0.0       0.0   0.0      0.0          0.0     0.0   0.0   
plunging       0.0       0.0   0.0      0.0          0.0     0.0   0.0   
tide           0.0       0.0   0.0      0.0          0.0     0.0   0.0   
violent        0.0       0.0   0.0      0.0          0.0     0.0   0.0   
stethoscope    0.0       0.0   0.0      0.0          0.0     0.0   0.0   
chairs         0.0       0.0   0.0      0.0          0.0     0.0   0.0   
work           0.0       0.0   0.0      0.0          0.0     0.0   0.0   
windows        0.0       0.0   0.0      0.0          0.0     0.0   0.0   
fact           0.0       0.0   0.0      0.0          0.0     0.0   0.0   
plainer        0.0       0.0   0.0      0.0          0.0     0.0   0.0   

             windows  fact  plainer  
couch            0.0   0.0      0.0  
plunging         0.0   0.0      0.0  
tide             0.0   0.0      0.0  
violent          0.0   0.0      0.0  
stethoscope      0.0   0.0      0.0  
chairs           0.0   0.0      0.0  
work             0.0   0.0      0.0  
windows          0.0   0.0      0.0  
fact             0.0   0.0      0.0  
plainer          0.0   0.0      0.0  

    \end{Verbatim}

    \begin{Verbatim}[commandchars=\\\{\}]
{\color{incolor}In [{\color{incolor}24}]:} \PY{c+c1}{\PYZsh{}\PYZsh{}\PYZsh{} SVD application}
         \PY{n}{la} \PY{o}{=} \PY{n}{np}\PY{o}{.}\PY{n}{linalg}
         \PY{n}{U}\PY{p}{,} \PY{n}{s}\PY{p}{,} \PY{n}{vh} \PY{o}{=} \PY{n}{la}\PY{o}{.}\PY{n}{svd}\PY{p}{(}\PY{n}{matrix}\PY{p}{,} \PY{n}{full\PYZus{}matrices}\PY{o}{=}\PY{k+kc}{False}\PY{p}{)}
         
         \PY{k}{for} \PY{n}{i} \PY{o+ow}{in} \PY{n+nb}{range}\PY{p}{(}\PY{n+nb}{len}\PY{p}{(}\PY{n}{matrix}\PY{p}{)}\PY{p}{)}\PY{p}{:}
             \PY{n}{plt}\PY{o}{.}\PY{n}{text}\PY{p}{(}\PY{n}{U}\PY{p}{[}\PY{n}{i}\PY{p}{,}\PY{l+m+mi}{0}\PY{p}{]}\PY{p}{,} \PY{n}{U}\PY{p}{[}\PY{n}{i}\PY{p}{,}\PY{l+m+mi}{1}\PY{p}{]}\PY{p}{,} \PY{n}{corpus}\PY{p}{[}\PY{n}{i}\PY{p}{]}\PY{p}{)}
\end{Verbatim}


    \begin{center}
    \adjustimage{max size={0.9\linewidth}{0.9\paperheight}}{svd_img.png}
    \end{center}
    
    \begin{center}
    \adjustimage{max size={1.2\linewidth}{0.9\paperheight}}{svd_img2.png}
    \end{center}
    
    \begin{center}
    \adjustimage{max size={1.2\linewidth}{0.9\paperheight}}{svd_img3.png}
    \end{center}
    { \hspace*{\fill} \\}
    

    % Add a bibliography block to the postdoc
    
    
    
    \end{document}
