
% Default to the notebook output style

    


% Inherit from the specified cell style.




    
\documentclass[11pt]{article}

    
    
    \usepackage[T1]{fontenc}
    % Nicer default font (+ math font) than Computer Modern for most use cases
    \usepackage{mathpazo}

    % Basic figure setup, for now with no caption control since it's done
    % automatically by Pandoc (which extracts ![](path) syntax from Markdown).
    \usepackage{graphicx}
    % We will generate all images so they have a width \maxwidth. This means
    % that they will get their normal width if they fit onto the page, but
    % are scaled down if they would overflow the margins.
    \makeatletter
    \def\maxwidth{\ifdim\Gin@nat@width>\linewidth\linewidth
    \else\Gin@nat@width\fi}
    \makeatother
    \let\Oldincludegraphics\includegraphics
    % Set max figure width to be 80% of text width, for now hardcoded.
    \renewcommand{\includegraphics}[1]{\Oldincludegraphics[width=.8\maxwidth]{#1}}
    % Ensure that by default, figures have no caption (until we provide a
    % proper Figure object with a Caption API and a way to capture that
    % in the conversion process - todo).
    \usepackage{caption}
    \DeclareCaptionLabelFormat{nolabel}{}
    \captionsetup{labelformat=nolabel}

    \usepackage{adjustbox} % Used to constrain images to a maximum size 
    \usepackage{xcolor} % Allow colors to be defined
    \usepackage{enumerate} % Needed for markdown enumerations to work
    \usepackage{geometry} % Used to adjust the document margins
    \usepackage{amsmath} % Equations
    \usepackage{amssymb} % Equations
    \usepackage{textcomp} % defines textquotesingle
    % Hack from http://tex.stackexchange.com/a/47451/13684:
    \AtBeginDocument{%
        \def\PYZsq{\textquotesingle}% Upright quotes in Pygmentized code
    }
    \usepackage{upquote} % Upright quotes for verbatim code
    \usepackage{eurosym} % defines \euro
    \usepackage[mathletters]{ucs} % Extended unicode (utf-8) support
    \usepackage[utf8x]{inputenc} % Allow utf-8 characters in the tex document
    \usepackage{fancyvrb} % verbatim replacement that allows latex
    \usepackage{grffile} % extends the file name processing of package graphics 
                         % to support a larger range 
    % The hyperref package gives us a pdf with properly built
    % internal navigation ('pdf bookmarks' for the table of contents,
    % internal cross-reference links, web links for URLs, etc.)
    \usepackage{hyperref}
    \usepackage{longtable} % longtable support required by pandoc >1.10
    \usepackage{booktabs}  % table support for pandoc > 1.12.2
    \usepackage[inline]{enumitem} % IRkernel/repr support (it uses the enumerate* environment)
    \usepackage[normalem]{ulem} % ulem is needed to support strikethroughs (\sout)
                                % normalem makes italics be italics, not underlines
    

    
    
    % Colors for the hyperref package
    \definecolor{urlcolor}{rgb}{0,.145,.698}
    \definecolor{linkcolor}{rgb}{.71,0.21,0.01}
    \definecolor{citecolor}{rgb}{.12,.54,.11}

    % ANSI colors
    \definecolor{ansi-black}{HTML}{3E424D}
    \definecolor{ansi-black-intense}{HTML}{282C36}
    \definecolor{ansi-red}{HTML}{E75C58}
    \definecolor{ansi-red-intense}{HTML}{B22B31}
    \definecolor{ansi-green}{HTML}{00A250}
    \definecolor{ansi-green-intense}{HTML}{007427}
    \definecolor{ansi-yellow}{HTML}{DDB62B}
    \definecolor{ansi-yellow-intense}{HTML}{B27D12}
    \definecolor{ansi-blue}{HTML}{208FFB}
    \definecolor{ansi-blue-intense}{HTML}{0065CA}
    \definecolor{ansi-magenta}{HTML}{D160C4}
    \definecolor{ansi-magenta-intense}{HTML}{A03196}
    \definecolor{ansi-cyan}{HTML}{60C6C8}
    \definecolor{ansi-cyan-intense}{HTML}{258F8F}
    \definecolor{ansi-white}{HTML}{C5C1B4}
    \definecolor{ansi-white-intense}{HTML}{A1A6B2}

    % commands and environments needed by pandoc snippets
    % extracted from the output of `pandoc -s`
    \providecommand{\tightlist}{%
      \setlength{\itemsep}{0pt}\setlength{\parskip}{0pt}}
    \DefineVerbatimEnvironment{Highlighting}{Verbatim}{commandchars=\\\{\}}
    % Add ',fontsize=\small' for more characters per line
    \newenvironment{Shaded}{}{}
    \newcommand{\KeywordTok}[1]{\textcolor[rgb]{0.00,0.44,0.13}{\textbf{{#1}}}}
    \newcommand{\DataTypeTok}[1]{\textcolor[rgb]{0.56,0.13,0.00}{{#1}}}
    \newcommand{\DecValTok}[1]{\textcolor[rgb]{0.25,0.63,0.44}{{#1}}}
    \newcommand{\BaseNTok}[1]{\textcolor[rgb]{0.25,0.63,0.44}{{#1}}}
    \newcommand{\FloatTok}[1]{\textcolor[rgb]{0.25,0.63,0.44}{{#1}}}
    \newcommand{\CharTok}[1]{\textcolor[rgb]{0.25,0.44,0.63}{{#1}}}
    \newcommand{\StringTok}[1]{\textcolor[rgb]{0.25,0.44,0.63}{{#1}}}
    \newcommand{\CommentTok}[1]{\textcolor[rgb]{0.38,0.63,0.69}{\textit{{#1}}}}
    \newcommand{\OtherTok}[1]{\textcolor[rgb]{0.00,0.44,0.13}{{#1}}}
    \newcommand{\AlertTok}[1]{\textcolor[rgb]{1.00,0.00,0.00}{\textbf{{#1}}}}
    \newcommand{\FunctionTok}[1]{\textcolor[rgb]{0.02,0.16,0.49}{{#1}}}
    \newcommand{\RegionMarkerTok}[1]{{#1}}
    \newcommand{\ErrorTok}[1]{\textcolor[rgb]{1.00,0.00,0.00}{\textbf{{#1}}}}
    \newcommand{\NormalTok}[1]{{#1}}
    
    % Additional commands for more recent versions of Pandoc
    \newcommand{\ConstantTok}[1]{\textcolor[rgb]{0.53,0.00,0.00}{{#1}}}
    \newcommand{\SpecialCharTok}[1]{\textcolor[rgb]{0.25,0.44,0.63}{{#1}}}
    \newcommand{\VerbatimStringTok}[1]{\textcolor[rgb]{0.25,0.44,0.63}{{#1}}}
    \newcommand{\SpecialStringTok}[1]{\textcolor[rgb]{0.73,0.40,0.53}{{#1}}}
    \newcommand{\ImportTok}[1]{{#1}}
    \newcommand{\DocumentationTok}[1]{\textcolor[rgb]{0.73,0.13,0.13}{\textit{{#1}}}}
    \newcommand{\AnnotationTok}[1]{\textcolor[rgb]{0.38,0.63,0.69}{\textbf{\textit{{#1}}}}}
    \newcommand{\CommentVarTok}[1]{\textcolor[rgb]{0.38,0.63,0.69}{\textbf{\textit{{#1}}}}}
    \newcommand{\VariableTok}[1]{\textcolor[rgb]{0.10,0.09,0.49}{{#1}}}
    \newcommand{\ControlFlowTok}[1]{\textcolor[rgb]{0.00,0.44,0.13}{\textbf{{#1}}}}
    \newcommand{\OperatorTok}[1]{\textcolor[rgb]{0.40,0.40,0.40}{{#1}}}
    \newcommand{\BuiltInTok}[1]{{#1}}
    \newcommand{\ExtensionTok}[1]{{#1}}
    \newcommand{\PreprocessorTok}[1]{\textcolor[rgb]{0.74,0.48,0.00}{{#1}}}
    \newcommand{\AttributeTok}[1]{\textcolor[rgb]{0.49,0.56,0.16}{{#1}}}
    \newcommand{\InformationTok}[1]{\textcolor[rgb]{0.38,0.63,0.69}{\textbf{\textit{{#1}}}}}
    \newcommand{\WarningTok}[1]{\textcolor[rgb]{0.38,0.63,0.69}{\textbf{\textit{{#1}}}}}
    
    
    % Define a nice break command that doesn't care if a line doesn't already
    % exist.
    \def\br{\hspace*{\fill} \\* }
    % Math Jax compatability definitions
    \def\gt{>}
    \def\lt{<}
    % Document parameters
    \title{Machine Learning: Code Example}
    
    \author{Fernando Tocantins }
    \date{January 2020}
    

    % Pygments definitions
    
\makeatletter
\def\PY@reset{\let\PY@it=\relax \let\PY@bf=\relax%
    \let\PY@ul=\relax \let\PY@tc=\relax%
    \let\PY@bc=\relax \let\PY@ff=\relax}
\def\PY@tok#1{\csname PY@tok@#1\endcsname}
\def\PY@toks#1+{\ifx\relax#1\empty\else%
    \PY@tok{#1}\expandafter\PY@toks\fi}
\def\PY@do#1{\PY@bc{\PY@tc{\PY@ul{%
    \PY@it{\PY@bf{\PY@ff{#1}}}}}}}
\def\PY#1#2{\PY@reset\PY@toks#1+\relax+\PY@do{#2}}

\expandafter\def\csname PY@tok@w\endcsname{\def\PY@tc##1{\textcolor[rgb]{0.73,0.73,0.73}{##1}}}
\expandafter\def\csname PY@tok@c\endcsname{\let\PY@it=\textit\def\PY@tc##1{\textcolor[rgb]{0.25,0.50,0.50}{##1}}}
\expandafter\def\csname PY@tok@cp\endcsname{\def\PY@tc##1{\textcolor[rgb]{0.74,0.48,0.00}{##1}}}
\expandafter\def\csname PY@tok@k\endcsname{\let\PY@bf=\textbf\def\PY@tc##1{\textcolor[rgb]{0.00,0.50,0.00}{##1}}}
\expandafter\def\csname PY@tok@kp\endcsname{\def\PY@tc##1{\textcolor[rgb]{0.00,0.50,0.00}{##1}}}
\expandafter\def\csname PY@tok@kt\endcsname{\def\PY@tc##1{\textcolor[rgb]{0.69,0.00,0.25}{##1}}}
\expandafter\def\csname PY@tok@o\endcsname{\def\PY@tc##1{\textcolor[rgb]{0.40,0.40,0.40}{##1}}}
\expandafter\def\csname PY@tok@ow\endcsname{\let\PY@bf=\textbf\def\PY@tc##1{\textcolor[rgb]{0.67,0.13,1.00}{##1}}}
\expandafter\def\csname PY@tok@nb\endcsname{\def\PY@tc##1{\textcolor[rgb]{0.00,0.50,0.00}{##1}}}
\expandafter\def\csname PY@tok@nf\endcsname{\def\PY@tc##1{\textcolor[rgb]{0.00,0.00,1.00}{##1}}}
\expandafter\def\csname PY@tok@nc\endcsname{\let\PY@bf=\textbf\def\PY@tc##1{\textcolor[rgb]{0.00,0.00,1.00}{##1}}}
\expandafter\def\csname PY@tok@nn\endcsname{\let\PY@bf=\textbf\def\PY@tc##1{\textcolor[rgb]{0.00,0.00,1.00}{##1}}}
\expandafter\def\csname PY@tok@ne\endcsname{\let\PY@bf=\textbf\def\PY@tc##1{\textcolor[rgb]{0.82,0.25,0.23}{##1}}}
\expandafter\def\csname PY@tok@nv\endcsname{\def\PY@tc##1{\textcolor[rgb]{0.10,0.09,0.49}{##1}}}
\expandafter\def\csname PY@tok@no\endcsname{\def\PY@tc##1{\textcolor[rgb]{0.53,0.00,0.00}{##1}}}
\expandafter\def\csname PY@tok@nl\endcsname{\def\PY@tc##1{\textcolor[rgb]{0.63,0.63,0.00}{##1}}}
\expandafter\def\csname PY@tok@ni\endcsname{\let\PY@bf=\textbf\def\PY@tc##1{\textcolor[rgb]{0.60,0.60,0.60}{##1}}}
\expandafter\def\csname PY@tok@na\endcsname{\def\PY@tc##1{\textcolor[rgb]{0.49,0.56,0.16}{##1}}}
\expandafter\def\csname PY@tok@nt\endcsname{\let\PY@bf=\textbf\def\PY@tc##1{\textcolor[rgb]{0.00,0.50,0.00}{##1}}}
\expandafter\def\csname PY@tok@nd\endcsname{\def\PY@tc##1{\textcolor[rgb]{0.67,0.13,1.00}{##1}}}
\expandafter\def\csname PY@tok@s\endcsname{\def\PY@tc##1{\textcolor[rgb]{0.73,0.13,0.13}{##1}}}
\expandafter\def\csname PY@tok@sd\endcsname{\let\PY@it=\textit\def\PY@tc##1{\textcolor[rgb]{0.73,0.13,0.13}{##1}}}
\expandafter\def\csname PY@tok@si\endcsname{\let\PY@bf=\textbf\def\PY@tc##1{\textcolor[rgb]{0.73,0.40,0.53}{##1}}}
\expandafter\def\csname PY@tok@se\endcsname{\let\PY@bf=\textbf\def\PY@tc##1{\textcolor[rgb]{0.73,0.40,0.13}{##1}}}
\expandafter\def\csname PY@tok@sr\endcsname{\def\PY@tc##1{\textcolor[rgb]{0.73,0.40,0.53}{##1}}}
\expandafter\def\csname PY@tok@ss\endcsname{\def\PY@tc##1{\textcolor[rgb]{0.10,0.09,0.49}{##1}}}
\expandafter\def\csname PY@tok@sx\endcsname{\def\PY@tc##1{\textcolor[rgb]{0.00,0.50,0.00}{##1}}}
\expandafter\def\csname PY@tok@m\endcsname{\def\PY@tc##1{\textcolor[rgb]{0.40,0.40,0.40}{##1}}}
\expandafter\def\csname PY@tok@gh\endcsname{\let\PY@bf=\textbf\def\PY@tc##1{\textcolor[rgb]{0.00,0.00,0.50}{##1}}}
\expandafter\def\csname PY@tok@gu\endcsname{\let\PY@bf=\textbf\def\PY@tc##1{\textcolor[rgb]{0.50,0.00,0.50}{##1}}}
\expandafter\def\csname PY@tok@gd\endcsname{\def\PY@tc##1{\textcolor[rgb]{0.63,0.00,0.00}{##1}}}
\expandafter\def\csname PY@tok@gi\endcsname{\def\PY@tc##1{\textcolor[rgb]{0.00,0.63,0.00}{##1}}}
\expandafter\def\csname PY@tok@gr\endcsname{\def\PY@tc##1{\textcolor[rgb]{1.00,0.00,0.00}{##1}}}
\expandafter\def\csname PY@tok@ge\endcsname{\let\PY@it=\textit}
\expandafter\def\csname PY@tok@gs\endcsname{\let\PY@bf=\textbf}
\expandafter\def\csname PY@tok@gp\endcsname{\let\PY@bf=\textbf\def\PY@tc##1{\textcolor[rgb]{0.00,0.00,0.50}{##1}}}
\expandafter\def\csname PY@tok@go\endcsname{\def\PY@tc##1{\textcolor[rgb]{0.53,0.53,0.53}{##1}}}
\expandafter\def\csname PY@tok@gt\endcsname{\def\PY@tc##1{\textcolor[rgb]{0.00,0.27,0.87}{##1}}}
\expandafter\def\csname PY@tok@err\endcsname{\def\PY@bc##1{\setlength{\fboxsep}{0pt}\fcolorbox[rgb]{1.00,0.00,0.00}{1,1,1}{\strut ##1}}}
\expandafter\def\csname PY@tok@kc\endcsname{\let\PY@bf=\textbf\def\PY@tc##1{\textcolor[rgb]{0.00,0.50,0.00}{##1}}}
\expandafter\def\csname PY@tok@kd\endcsname{\let\PY@bf=\textbf\def\PY@tc##1{\textcolor[rgb]{0.00,0.50,0.00}{##1}}}
\expandafter\def\csname PY@tok@kn\endcsname{\let\PY@bf=\textbf\def\PY@tc##1{\textcolor[rgb]{0.00,0.50,0.00}{##1}}}
\expandafter\def\csname PY@tok@kr\endcsname{\let\PY@bf=\textbf\def\PY@tc##1{\textcolor[rgb]{0.00,0.50,0.00}{##1}}}
\expandafter\def\csname PY@tok@bp\endcsname{\def\PY@tc##1{\textcolor[rgb]{0.00,0.50,0.00}{##1}}}
\expandafter\def\csname PY@tok@fm\endcsname{\def\PY@tc##1{\textcolor[rgb]{0.00,0.00,1.00}{##1}}}
\expandafter\def\csname PY@tok@vc\endcsname{\def\PY@tc##1{\textcolor[rgb]{0.10,0.09,0.49}{##1}}}
\expandafter\def\csname PY@tok@vg\endcsname{\def\PY@tc##1{\textcolor[rgb]{0.10,0.09,0.49}{##1}}}
\expandafter\def\csname PY@tok@vi\endcsname{\def\PY@tc##1{\textcolor[rgb]{0.10,0.09,0.49}{##1}}}
\expandafter\def\csname PY@tok@vm\endcsname{\def\PY@tc##1{\textcolor[rgb]{0.10,0.09,0.49}{##1}}}
\expandafter\def\csname PY@tok@sa\endcsname{\def\PY@tc##1{\textcolor[rgb]{0.73,0.13,0.13}{##1}}}
\expandafter\def\csname PY@tok@sb\endcsname{\def\PY@tc##1{\textcolor[rgb]{0.73,0.13,0.13}{##1}}}
\expandafter\def\csname PY@tok@sc\endcsname{\def\PY@tc##1{\textcolor[rgb]{0.73,0.13,0.13}{##1}}}
\expandafter\def\csname PY@tok@dl\endcsname{\def\PY@tc##1{\textcolor[rgb]{0.73,0.13,0.13}{##1}}}
\expandafter\def\csname PY@tok@s2\endcsname{\def\PY@tc##1{\textcolor[rgb]{0.73,0.13,0.13}{##1}}}
\expandafter\def\csname PY@tok@sh\endcsname{\def\PY@tc##1{\textcolor[rgb]{0.73,0.13,0.13}{##1}}}
\expandafter\def\csname PY@tok@s1\endcsname{\def\PY@tc##1{\textcolor[rgb]{0.73,0.13,0.13}{##1}}}
\expandafter\def\csname PY@tok@mb\endcsname{\def\PY@tc##1{\textcolor[rgb]{0.40,0.40,0.40}{##1}}}
\expandafter\def\csname PY@tok@mf\endcsname{\def\PY@tc##1{\textcolor[rgb]{0.40,0.40,0.40}{##1}}}
\expandafter\def\csname PY@tok@mh\endcsname{\def\PY@tc##1{\textcolor[rgb]{0.40,0.40,0.40}{##1}}}
\expandafter\def\csname PY@tok@mi\endcsname{\def\PY@tc##1{\textcolor[rgb]{0.40,0.40,0.40}{##1}}}
\expandafter\def\csname PY@tok@il\endcsname{\def\PY@tc##1{\textcolor[rgb]{0.40,0.40,0.40}{##1}}}
\expandafter\def\csname PY@tok@mo\endcsname{\def\PY@tc##1{\textcolor[rgb]{0.40,0.40,0.40}{##1}}}
\expandafter\def\csname PY@tok@ch\endcsname{\let\PY@it=\textit\def\PY@tc##1{\textcolor[rgb]{0.25,0.50,0.50}{##1}}}
\expandafter\def\csname PY@tok@cm\endcsname{\let\PY@it=\textit\def\PY@tc##1{\textcolor[rgb]{0.25,0.50,0.50}{##1}}}
\expandafter\def\csname PY@tok@cpf\endcsname{\let\PY@it=\textit\def\PY@tc##1{\textcolor[rgb]{0.25,0.50,0.50}{##1}}}
\expandafter\def\csname PY@tok@c1\endcsname{\let\PY@it=\textit\def\PY@tc##1{\textcolor[rgb]{0.25,0.50,0.50}{##1}}}
\expandafter\def\csname PY@tok@cs\endcsname{\let\PY@it=\textit\def\PY@tc##1{\textcolor[rgb]{0.25,0.50,0.50}{##1}}}

\def\PYZbs{\char`\\}
\def\PYZus{\char`\_}
\def\PYZob{\char`\{}
\def\PYZcb{\char`\}}
\def\PYZca{\char`\^}
\def\PYZam{\char`\&}
\def\PYZlt{\char`\<}
\def\PYZgt{\char`\>}
\def\PYZsh{\char`\#}
\def\PYZpc{\char`\%}
\def\PYZdl{\char`\$}
\def\PYZhy{\char`\-}
\def\PYZsq{\char`\'}
\def\PYZdq{\char`\"}
\def\PYZti{\char`\~}
% for compatibility with earlier versions
\def\PYZat{@}
\def\PYZlb{[}
\def\PYZrb{]}
\makeatother


    % Exact colors from NB
    \definecolor{incolor}{rgb}{0.0, 0.0, 0.5}
    \definecolor{outcolor}{rgb}{0.545, 0.0, 0.0}



    
    % Prevent overflowing lines due to hard-to-break entities
    \sloppy 
    % Setup hyperref package
    \hypersetup{
      breaklinks=true,  % so long urls are correctly broken across lines
      colorlinks=true,
      urlcolor=urlcolor,
      linkcolor=linkcolor,
      citecolor=citecolor,
      }
    % Slightly bigger margins than the latex defaults
    
    \geometry{verbose,tmargin=1in,bmargin=1in,lmargin=1in,rmargin=1in}
    
    

    \begin{document}
    
    
    \maketitle
    
    

    
    \section{Multiple Linear Regression with categorical value in
dataset}\label{multiple-linear-regression-with-categorical-value-in-dataset}

    \subparagraph{Libraries}\label{libraries}

\begin{itemize}
\tightlist
\item
  Pandas https://pandas.pydata.org/, a tool for easy data structring,
  reading and analyses. It will be used for imprting and reading the
  document
\item
  ScikitLearn https://scikit-learn.org/stable/, a library of tools for
  predictive analyses. It will be used for the importing of Linear
  Regression module
\item
  MatplotLib https://matplotlib.org/, for graph plotting.
\item
  Numpy https://numpy.org/, almost a standard library for scientific
  computing, has many resources for data manipulation.
\end{itemize}

    \begin{quote}
This example has as objective build a Multiple Linear Regression with a
categorical value into the dataset. Categorical values are non-
numerical parametersthat must be transformed into numerical data for the
implementation in the model. The dataset below shows parameters of R\&D
Spend, Administration cost, Marketing Spend, Profit and State of
residence from fictitious companies in USA. The objective of this code
is to buid a linear regression prediction model for the company's Profit
using the other parameters as input values.
\end{quote}

    \begin{Verbatim}[commandchars=\\\{\}]
{\color{incolor}In [{\color{incolor}10}]:} \PY{c+c1}{\PYZsh{} Importing the libraries}
         \PY{k+kn}{import} \PY{n+nn}{numpy} \PY{k}{as} \PY{n+nn}{np}
         \PY{k+kn}{import} \PY{n+nn}{matplotlib}\PY{n+nn}{.}\PY{n+nn}{pyplot} \PY{k}{as} \PY{n+nn}{plt}
         \PY{k+kn}{import} \PY{n+nn}{pandas} \PY{k}{as} \PY{n+nn}{pd}
         
         \PY{c+c1}{\PYZsh{} Importing the dataset}
         \PY{n}{df} \PY{o}{=} \PY{n}{pd}\PY{o}{.}\PY{n}{read\PYZus{}csv}\PY{p}{(}\PY{l+s+s1}{\PYZsq{}}\PY{l+s+s1}{50\PYZus{}Startups.csv}\PY{l+s+s1}{\PYZsq{}}\PY{p}{)}
         
         \PY{n+nb}{print}\PY{p}{(}\PY{n}{df}\PY{p}{)}
\end{Verbatim}


    \begin{Verbatim}[commandchars=\\\{\}]
    R\&D Spend  Administration  Marketing Spend       State     Profit
0   165349.20       136897.80        471784.10    New York  192261.83
1   162597.70       151377.59        443898.53  California  191792.06
2   153441.51       101145.55        407934.54     Florida  191050.39
3   144372.41       118671.85        383199.62    New York  182901.99
4   142107.34        91391.77        366168.42     Florida  166187.94
5   131876.90        99814.71        362861.36    New York  156991.12
6   134615.46       147198.87        127716.82  California  156122.51
7   130298.13       145530.06        323876.68     Florida  155752.60
8   120542.52       148718.95        311613.29    New York  152211.77
9   123334.88       108679.17        304981.62  California  149759.96
10  101913.08       110594.11        229160.95     Florida  146121.95
11  100671.96        91790.61        249744.55  California  144259.40
12   93863.75       127320.38        249839.44     Florida  141585.52
13   91992.39       135495.07        252664.93  California  134307.35
14  119943.24       156547.42        256512.92     Florida  132602.65
15  114523.61       122616.84        261776.23    New York  129917.04
16   78013.11       121597.55        264346.06  California  126992.93
17   94657.16       145077.58        282574.31    New York  125370.37
18   91749.16       114175.79        294919.57     Florida  124266.90
19   86419.70       153514.11             0.00    New York  122776.86
20   76253.86       113867.30        298664.47  California  118474.03
21   78389.47       153773.43        299737.29    New York  111313.02
22   73994.56       122782.75        303319.26     Florida  110352.25
23   67532.53       105751.03        304768.73     Florida  108733.99
24   77044.01        99281.34        140574.81    New York  108552.04
25   64664.71       139553.16        137962.62  California  107404.34
26   75328.87       144135.98        134050.07     Florida  105733.54
27   72107.60       127864.55        353183.81    New York  105008.31
28   66051.52       182645.56        118148.20     Florida  103282.38
29   65605.48       153032.06        107138.38    New York  101004.64
30   61994.48       115641.28         91131.24     Florida   99937.59
31   61136.38       152701.92         88218.23    New York   97483.56
32   63408.86       129219.61         46085.25  California   97427.84
33   55493.95       103057.49        214634.81     Florida   96778.92
34   46426.07       157693.92        210797.67  California   96712.80
35   46014.02        85047.44        205517.64    New York   96479.51
36   28663.76       127056.21        201126.82     Florida   90708.19
37   44069.95        51283.14        197029.42  California   89949.14
38   20229.59        65947.93        185265.10    New York   81229.06
39   38558.51        82982.09        174999.30  California   81005.76
40   28754.33       118546.05        172795.67  California   78239.91
41   27892.92        84710.77        164470.71     Florida   77798.83
42   23640.93        96189.63        148001.11  California   71498.49
43   15505.73       127382.30         35534.17    New York   69758.98
44   22177.74       154806.14         28334.72  California   65200.33
45    1000.23       124153.04          1903.93    New York   64926.08
46    1315.46       115816.21        297114.46     Florida   49490.75
47       0.00       135426.92             0.00  California   42559.73
48     542.05        51743.15             0.00    New York   35673.41
49       0.00       116983.80         45173.06  California   14681.40

    \end{Verbatim}

    \begin{Verbatim}[commandchars=\\\{\}]
{\color{incolor}In [{\color{incolor}11}]:} \PY{c+c1}{\PYZsh{} Setting R\PYZam{}D Spend, Administration cost, Marketing Spend and State as input values}
         \PY{n}{X} \PY{o}{=} \PY{n}{df}\PY{o}{.}\PY{n}{iloc}\PY{p}{[}\PY{p}{:}\PY{p}{,} \PY{p}{:}\PY{o}{\PYZhy{}}\PY{l+m+mi}{1}\PY{p}{]}\PY{o}{.}\PY{n}{values}
         
         \PY{c+c1}{\PYZsh{} Setting Profit as output value}
         \PY{n}{y} \PY{o}{=} \PY{n}{df}\PY{o}{.}\PY{n}{iloc}\PY{p}{[}\PY{p}{:}\PY{p}{,} \PY{l+m+mi}{4}\PY{p}{]}\PY{o}{.}\PY{n}{values}
         
         \PY{n+nb}{print}\PY{p}{(}\PY{n}{X}\PY{p}{)}
\end{Verbatim}


    \begin{Verbatim}[commandchars=\\\{\}]
[[165349.2 136897.8 471784.1 'New York']
 [162597.7 151377.59 443898.53 'California']
 [153441.51 101145.55 407934.54 'Florida']
 [144372.41 118671.85 383199.62 'New York']
 [142107.34 91391.77 366168.42 'Florida']
 [131876.9 99814.71 362861.36 'New York']
 [134615.46 147198.87 127716.82 'California']
 [130298.13 145530.06 323876.68 'Florida']
 [120542.52 148718.95 311613.29 'New York']
 [123334.88 108679.17 304981.62 'California']
 [101913.08 110594.11 229160.95 'Florida']
 [100671.96 91790.61 249744.55 'California']
 [93863.75 127320.38 249839.44 'Florida']
 [91992.39 135495.07 252664.93 'California']
 [119943.24 156547.42 256512.92 'Florida']
 [114523.61 122616.84 261776.23 'New York']
 [78013.11 121597.55 264346.06 'California']
 [94657.16 145077.58 282574.31 'New York']
 [91749.16 114175.79 294919.57 'Florida']
 [86419.7 153514.11 0.0 'New York']
 [76253.86 113867.3 298664.47 'California']
 [78389.47 153773.43 299737.29 'New York']
 [73994.56 122782.75 303319.26 'Florida']
 [67532.53 105751.03 304768.73 'Florida']
 [77044.01 99281.34 140574.81 'New York']
 [64664.71 139553.16 137962.62 'California']
 [75328.87 144135.98 134050.07 'Florida']
 [72107.6 127864.55 353183.81 'New York']
 [66051.52 182645.56 118148.2 'Florida']
 [65605.48 153032.06 107138.38 'New York']
 [61994.48 115641.28 91131.24 'Florida']
 [61136.38 152701.92 88218.23 'New York']
 [63408.86 129219.61 46085.25 'California']
 [55493.95 103057.49 214634.81 'Florida']
 [46426.07 157693.92 210797.67 'California']
 [46014.02 85047.44 205517.64 'New York']
 [28663.76 127056.21 201126.82 'Florida']
 [44069.95 51283.14 197029.42 'California']
 [20229.59 65947.93 185265.1 'New York']
 [38558.51 82982.09 174999.3 'California']
 [28754.33 118546.05 172795.67 'California']
 [27892.92 84710.77 164470.71 'Florida']
 [23640.93 96189.63 148001.11 'California']
 [15505.73 127382.3 35534.17 'New York']
 [22177.74 154806.14 28334.72 'California']
 [1000.23 124153.04 1903.93 'New York']
 [1315.46 115816.21 297114.46 'Florida']
 [0.0 135426.92 0.0 'California']
 [542.05 51743.15 0.0 'New York']
 [0.0 116983.8 45173.06 'California']]

    \end{Verbatim}

    \begin{Verbatim}[commandchars=\\\{\}]
{\color{incolor}In [{\color{incolor}12}]:} \PY{n+nb}{print}\PY{p}{(}\PY{n}{y}\PY{p}{)}
\end{Verbatim}


    \begin{Verbatim}[commandchars=\\\{\}]
[192261.83 191792.06 191050.39 182901.99 166187.94 156991.12 156122.51
 155752.6  152211.77 149759.96 146121.95 144259.4  141585.52 134307.35
 132602.65 129917.04 126992.93 125370.37 124266.9  122776.86 118474.03
 111313.02 110352.25 108733.99 108552.04 107404.34 105733.54 105008.31
 103282.38 101004.64  99937.59  97483.56  97427.84  96778.92  96712.8
  96479.51  90708.19  89949.14  81229.06  81005.76  78239.91  77798.83
  71498.49  69758.98  65200.33  64926.08  49490.75  42559.73  35673.41
  14681.4 ]

    \end{Verbatim}

    \begin{quote}
The categorical data is divided into three different values (New York,
Florida, California). This values must be represented into numerical
values for the correct implementation in the model. The LabelEncoder and
OneHotEncoder from Scikit Learn library are the perfect tools for this
job. LabelEncoder will transform the categorical value in numerical
value by counting the number of diferent itens in the presented column
of categorical data, for example, if there are names, as above, it will
convert "New York" into 0, "Florida" into 1 and "California" into 2.
This is the first step when dealing with categorical features.
\end{quote}

    \begin{Verbatim}[commandchars=\\\{\}]
{\color{incolor}In [{\color{incolor}14}]:} \PY{c+c1}{\PYZsh{} Encoding the categorical data}
         \PY{k+kn}{from} \PY{n+nn}{sklearn}\PY{n+nn}{.}\PY{n+nn}{preprocessing} \PY{k}{import} \PY{n}{LabelEncoder}\PY{p}{,} \PY{n}{OneHotEncoder}
         \PY{n}{labelencoder} \PY{o}{=} \PY{n}{LabelEncoder}\PY{p}{(}\PY{p}{)}
         \PY{n}{X}\PY{p}{[}\PY{p}{:}\PY{p}{,} \PY{l+m+mi}{3}\PY{p}{]} \PY{o}{=} \PY{n}{labelencoder}\PY{o}{.}\PY{n}{fit\PYZus{}transform}\PY{p}{(}\PY{n}{X}\PY{p}{[}\PY{p}{:}\PY{p}{,} \PY{l+m+mi}{3}\PY{p}{]}\PY{p}{)}
         
         \PY{n+nb}{print}\PY{p}{(}\PY{n}{X}\PY{p}{[}\PY{p}{:}\PY{p}{,}\PY{l+m+mi}{3}\PY{p}{]}\PY{p}{)}
\end{Verbatim}


    \begin{Verbatim}[commandchars=\\\{\}]
[2 0 1 2 1 2 0 1 2 0 1 0 1 0 1 2 0 2 1 2 0 2 1 1 2 0 1 2 1 2 1 2 0 1 0 2 1
 0 2 0 0 1 0 2 0 2 1 0 2 0]

    \end{Verbatim}

    \begin{quote}
As a second step, it is useful for the model to break the different
weight in the transformed categorical feature that comes naturaly when
applied LabelEncoder. For example, "New York" wont has the same
relevance as "California", once it has the value of 0 and "California"
has 2. We just want that this variables to represent different
conditions, but not different weights. Hence, we must apply
OneHotEncoder as done below.
\end{quote}

    \begin{Verbatim}[commandchars=\\\{\}]
{\color{incolor}In [{\color{incolor}15}]:} \PY{n}{onehotencoder} \PY{o}{=} \PY{n}{OneHotEncoder}\PY{p}{(}\PY{n}{categorical\PYZus{}features} \PY{o}{=} \PY{p}{[}\PY{l+m+mi}{3}\PY{p}{]}\PY{p}{)}
         \PY{n}{X} \PY{o}{=} \PY{n}{onehotencoder}\PY{o}{.}\PY{n}{fit\PYZus{}transform}\PY{p}{(}\PY{n}{X}\PY{p}{)}\PY{o}{.}\PY{n}{toarray}\PY{p}{(}\PY{p}{)}
\end{Verbatim}


    \begin{Verbatim}[commandchars=\\\{\}]
{\color{incolor}In [{\color{incolor}16}]:} \PY{c+c1}{\PYZsh{} Excluding one line (taking out the dummy variable)}
         \PY{n}{X} \PY{o}{=} \PY{n}{X}\PY{p}{[}\PY{p}{:}\PY{p}{,} \PY{l+m+mi}{1}\PY{p}{:}\PY{p}{]}
\end{Verbatim}


    \begin{quote}
The dummy variable is a unwanted consequence for OneHotEncoder usage. It
consists in a unnecessary column added into the dataset and must be
excluded as done above.
\end{quote}

    \begin{Verbatim}[commandchars=\\\{\}]
{\color{incolor}In [{\color{incolor}17}]:} \PY{c+c1}{\PYZsh{} Selecting the Training set and the Test set}
         \PY{k+kn}{from} \PY{n+nn}{sklearn}\PY{n+nn}{.}\PY{n+nn}{cross\PYZus{}validation} \PY{k}{import} \PY{n}{train\PYZus{}test\PYZus{}split}
         \PY{n}{X\PYZus{}train}\PY{p}{,} \PY{n}{X\PYZus{}test}\PY{p}{,} \PY{n}{y\PYZus{}train}\PY{p}{,} \PY{n}{y\PYZus{}test} \PY{o}{=} \PY{n}{train\PYZus{}test\PYZus{}split}\PY{p}{(}\PY{n}{X}\PY{p}{,} \PY{n}{y}\PY{p}{,} \PY{n}{test\PYZus{}size} \PY{o}{=} \PY{l+m+mf}{0.2}\PY{p}{,} \PY{n}{random\PYZus{}state} \PY{o}{=} \PY{l+m+mi}{0}\PY{p}{)}
         
         \PY{n+nb}{print}\PY{p}{(}\PY{l+s+s2}{\PYZdq{}}\PY{l+s+s2}{Total:}\PY{l+s+s2}{\PYZdq{}}\PY{p}{,}\PY{n+nb}{len}\PY{p}{(}\PY{n}{X}\PY{p}{)}\PY{p}{,} \PY{l+s+s2}{\PYZdq{}}\PY{l+s+s2}{ Teste: }\PY{l+s+s2}{\PYZdq{}}\PY{p}{,} \PY{n+nb}{len}\PY{p}{(}\PY{n}{X\PYZus{}test}\PY{p}{)}\PY{p}{,}\PY{l+s+s2}{\PYZdq{}}\PY{l+s+s2}{ Treino: }\PY{l+s+s2}{\PYZdq{}}\PY{p}{,} \PY{n+nb}{len}\PY{p}{(}\PY{n}{X\PYZus{}train}\PY{p}{)}\PY{p}{)}
\end{Verbatim}


    \begin{Verbatim}[commandchars=\\\{\}]
Total: 50  Teste:  10  Treino:  40

    \end{Verbatim}

    \begin{Verbatim}[commandchars=\\\{\}]
C:\textbackslash{}Users\textbackslash{}ftoca\textbackslash{}AppData\textbackslash{}Local\textbackslash{}Continuum\textbackslash{}anaconda3\textbackslash{}lib\textbackslash{}site-packages\textbackslash{}sklearn\textbackslash{}cross\_validation.py:41: DeprecationWarning: This module was deprecated in version 0.18 in favor of the model\_selection module into which all the refactored classes and functions are moved. Also note that the interface of the new CV iterators are different from that of this module. This module will be removed in 0.20.
  "This module will be removed in 0.20.", DeprecationWarning)

    \end{Verbatim}

    \begin{quote}
Linear Regression Model - Skcikit Learn
https://scikit-learn.org/stable/modules/generated/sklearn.linear\_model.LinearRegression.html
\end{quote}

    \begin{Verbatim}[commandchars=\\\{\}]
{\color{incolor}In [{\color{incolor}22}]:} \PY{c+c1}{\PYZsh{} Fitting the modeln to the Training set}
         \PY{k+kn}{from} \PY{n+nn}{sklearn}\PY{n+nn}{.}\PY{n+nn}{linear\PYZus{}model} \PY{k}{import} \PY{n}{LinearRegression}
         \PY{n}{regressor} \PY{o}{=} \PY{n}{LinearRegression}\PY{p}{(}\PY{p}{)}
         \PY{n}{regressor}\PY{o}{.}\PY{n}{fit}\PY{p}{(}\PY{n}{X\PYZus{}train}\PY{p}{,} \PY{n}{y\PYZus{}train}\PY{p}{)}
         
         \PY{c+c1}{\PYZsh{} Predicting the Test set}
         \PY{n}{y\PYZus{}pred} \PY{o}{=} \PY{n}{regressor}\PY{o}{.}\PY{n}{predict}\PY{p}{(}\PY{n}{X\PYZus{}test}\PY{p}{)}
         
         \PY{n+nb}{print}\PY{p}{(}\PY{n}{X\PYZus{}test}\PY{p}{,} \PY{l+s+s2}{\PYZdq{}}\PY{l+s+se}{\PYZbs{}n}\PY{l+s+s2}{\PYZdq{}}\PY{p}{)}
         \PY{n+nb}{print}\PY{p}{(}\PY{n}{y\PYZus{}pred}\PY{p}{)}
\end{Verbatim}


    \begin{Verbatim}[commandchars=\\\{\}]
[[1.0000000e+00 0.0000000e+00 6.6051520e+04 1.8264556e+05 1.1814820e+05]
 [0.0000000e+00 0.0000000e+00 1.0067196e+05 9.1790610e+04 2.4974455e+05]
 [1.0000000e+00 0.0000000e+00 1.0191308e+05 1.1059411e+05 2.2916095e+05]
 [1.0000000e+00 0.0000000e+00 2.7892920e+04 8.4710770e+04 1.6447071e+05]
 [1.0000000e+00 0.0000000e+00 1.5344151e+05 1.0114555e+05 4.0793454e+05]
 [0.0000000e+00 1.0000000e+00 7.2107600e+04 1.2786455e+05 3.5318381e+05]
 [0.0000000e+00 1.0000000e+00 2.0229590e+04 6.5947930e+04 1.8526510e+05]
 [0.0000000e+00 1.0000000e+00 6.1136380e+04 1.5270192e+05 8.8218230e+04]
 [1.0000000e+00 0.0000000e+00 7.3994560e+04 1.2278275e+05 3.0331926e+05]
 [1.0000000e+00 0.0000000e+00 1.4210734e+05 9.1391770e+04 3.6616842e+05]] 

[103015.20159796 132582.27760815 132447.73845175  71976.09851258
 178537.48221056 116161.24230166  67851.69209676  98791.73374687
 113969.43533013 167921.06569551]

    \end{Verbatim}

    \begin{Verbatim}[commandchars=\\\{\}]
{\color{incolor}In [{\color{incolor}23}]:} \PY{c+c1}{\PYZsh{} 10 first values of X }
         \PY{n+nb}{print}\PY{p}{(}\PY{n}{X}\PY{p}{[}\PY{p}{:}\PY{l+m+mi}{10}\PY{p}{,}\PY{p}{:}\PY{p}{]}\PY{p}{)}
\end{Verbatim}


    \begin{Verbatim}[commandchars=\\\{\}]
[[0.0000000e+00 1.0000000e+00 1.6534920e+05 1.3689780e+05 4.7178410e+05]
 [0.0000000e+00 0.0000000e+00 1.6259770e+05 1.5137759e+05 4.4389853e+05]
 [1.0000000e+00 0.0000000e+00 1.5344151e+05 1.0114555e+05 4.0793454e+05]
 [0.0000000e+00 1.0000000e+00 1.4437241e+05 1.1867185e+05 3.8319962e+05]
 [1.0000000e+00 0.0000000e+00 1.4210734e+05 9.1391770e+04 3.6616842e+05]
 [0.0000000e+00 1.0000000e+00 1.3187690e+05 9.9814710e+04 3.6286136e+05]
 [0.0000000e+00 0.0000000e+00 1.3461546e+05 1.4719887e+05 1.2771682e+05]
 [1.0000000e+00 0.0000000e+00 1.3029813e+05 1.4553006e+05 3.2387668e+05]
 [0.0000000e+00 1.0000000e+00 1.2054252e+05 1.4871895e+05 3.1161329e+05]
 [0.0000000e+00 0.0000000e+00 1.2333488e+05 1.0867917e+05 3.0498162e+05]]

    \end{Verbatim}

    \begin{Verbatim}[commandchars=\\\{\}]
{\color{incolor}In [{\color{incolor}24}]:} \PY{c+c1}{\PYZsh{} 10 First values of y}
         \PY{n+nb}{print}\PY{p}{(}\PY{n}{y}\PY{p}{[}\PY{p}{:}\PY{l+m+mi}{10}\PY{p}{]}\PY{p}{)}
\end{Verbatim}


    \begin{Verbatim}[commandchars=\\\{\}]
[192261.83 191792.06 191050.39 182901.99 166187.94 156991.12 156122.51
 155752.6  152211.77 149759.96]

    \end{Verbatim}

    \begin{Verbatim}[commandchars=\\\{\}]
{\color{incolor}In [{\color{incolor}25}]:} \PY{c+c1}{\PYZsh{} Prediction using the 10 first values of X}
         \PY{n}{y\PYZus{}predicted} \PY{o}{=} \PY{n}{regressor}\PY{o}{.}\PY{n}{predict}\PY{p}{(}\PY{n}{X}\PY{p}{[}\PY{p}{:}\PY{l+m+mi}{10}\PY{p}{,}\PY{p}{:}\PY{p}{]}\PY{p}{)}
         \PY{n+nb}{print}\PY{p}{(}\PY{n}{y\PYZus{}predicted}\PY{p}{)}
\end{Verbatim}


    \begin{Verbatim}[commandchars=\\\{\}]
[192919.57537463 189547.28196893 178537.48221056 172852.28336638
 167921.06569551 161822.7226407  156191.10124355 159019.04337114
 152787.95464278 152688.90677959]

    \end{Verbatim}


    % Add a bibliography block to the postdoc
    
    
    
    \end{document}
